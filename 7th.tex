\documentclass{article}  
\begin{document}  
\title{Study Git and LaTeX}  
\author{Donglin Yuan}  
\maketitle  
\section{Basic Grammar}  
\begin{enumerate}  
    \item Basic  
    \begin{itemize}  
        \item[1] \texttt{git config --global user.name "enter your name"} \\
        This command sets your user name.  
        \item[2] \texttt{git config --global user.email "enter your email"} \\
        Similar to above.  
        \item[3] \texttt{ssh-keygen -t rsa -C "your email"} \\
        It is worth mentioning that your computer must have SSH enabled to connect successfully.  
        \item[4] \texttt{ssh -T git@github.com} \\
        This command tests whether you have connected to github.com.  
    \end{itemize}  
    
    \item Advanced a little  
    \begin{itemize}  
        \item[5] \texttt{git init} \\
        This command initializes a folder.  
        \item[6] \texttt{git clone git@github.com:user\_name/repositories\_name.git} \\
        Put the repositories into your own folder on the computer.  
        \item[7] \texttt{git add foldername} or \texttt{git add filename} \\
        Adds these to the staging area.  
        \item[8] \texttt{git commit -m "annotation"} \\
        This command creates a commit with a message.  
        \item[9] \texttt{git push origin (master/newBranch\_name)} \\
        Push the file to the repositories on github.com.  
    \end{itemize}  
    \item Advanced more maybe  
    \begin{itemize}  
        \item[10] \texttt{git log} \\
        Look up history.  
        \item[11] \texttt{git reset --hard "version\_number"} \\
        This resets the version.  
    \end{itemize}  
    \item Branch Commands  
    \begin{itemize}  
        \item[12] \texttt{git branch "newBranchName"} \\
        Creates a new branch.  
        \item[13] \texttt{git branch -v} \\
        To know how many branches exist.  
        \item[14] \texttt{git checkout "branchName"} \\
        To switch to another branch.  
        \item[15] \texttt{git merge "mergedBranchName"} \\
        Merges the specified branch into your current branch.  
        \item[16] \texttt{git branch -a} \\
        Shows local and remote branches.  
    \end{itemize}  
    \item Something interesting  
    \begin{itemize}  
        \item[17] \texttt{git diff --shortstat "@{0 day ago}"} \\
        Shows how much work you've done.  
        \item[18] \texttt{git checkout .} \\
        Restores all files from the staging area to the working directory.  
        \item[19] \texttt{git checkout [file]} \\
        Restores this specific file from the staging area to the working directory.  
        \item[20] \texttt{git push [remote] [branch]} \\
        Pushes the branch to the remote repository.  
        \item[21] \texttt{git reflog} \\
        Shows your recent commit history.  
        \item[22] \texttt{git diff} \\
        Shows differences between the working directory and the staging area.  
        \item[23] \texttt{git status} \\
        Shows which files are modified.  
        \item[24] \texttt{git log --stat} \\
        Shows the history of commits and which files were modified each time.  
    \end{itemize}  
\end{enumerate}
\end{document} 