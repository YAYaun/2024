\documentclass{article}
\usepackage{xcolor}
\usepackage{graphicx}
\usepackage[UTF8]{ctex}
\begin{document}
\title{中国海洋大学实验报告}
\maketitle
\title{shell,vim,数据整理的学习\\https://github.com/YAYaun/2024\\}
\author{袁东霖}
\maketitle
\section{\textcolor{blue}{Creating and Navigating Directories}}
\begin{itemize}
    \item \texttt{mkdir -p "name"}:创建并打开 \texttt{name},如果没有父目录则一并创建。
    \item \texttt{cd "name"}:进入 \texttt{name} 目录。
\end{itemize}

\section{\textcolor{blue}{Viewing Help Documentation}}
\begin{itemize}
    \item \texttt{man 函数名}:查看函数的帮助文档。
\end{itemize}

\section{\textcolor{blue}{Viewing Files(课后习题}}
\begin{itemize}
    \item \texttt{ls -a}:查看所有文件,包含隐藏文件。
    \item \texttt{ls -lah}:以长文件形式显示,以人类可读的形式显示。
    \item \texttt{ls -lt}:显示最新修改的文件。
    \item \texttt{ls -lu}:显示最新访问的文件。
\end{itemize}

\section{\textcolor{blue}{Battery Information}}
\begin{itemize}
    \item \texttt{cat /sys/class/power\_supply/BAT0/capacity}:获取电池信息。
\end{itemize}

\section{\textcolor{blue}{Marco and Polo Functions(课后习题}}
\begin{itemize}
    \item \texttt{marco()\{ export marcoPath=\$(pwd); \}}:定义 \texttt{marco} 函数,保存当前路径。
    \item \texttt{polo()\{ cd "\$marcoPath"; \}}:定义 \texttt{polo} 函数,返回 \texttt{marco} 保存的路径。
\end{itemize}

\section{\textcolor{blue}{Vim Commands}}
\begin{itemize}
\begin{figure}[htbp]
    \centering
    \includegraphics[scale=0.2]{picturevim.png}
    \caption{Elliptic Paraboloid}
    \end{figure} 
    \item \texttt{i}:切换到输入模式。
    \item \texttt{a}:切换到插入模式。
    \item \texttt{x}:删除当前光标所在字符。
    \item \texttt{u}:撤销上一次操作。
    \item \texttt{ctrl+r}:重做上一次的操作。
    \item \texttt{:wq}:保存并退出。
    \item \texttt{\textbackslash sudo vim 文件名}:以管理员权限打开文件。
    \item \texttt{:s/foo/bar/g}:在整个文件把 \texttt{foo} 改成 \texttt{bar}。
    
\end{itemize}
\section{\textcolor{blue}{Regular expression}}
\begin{itemize}
  \item *:匹配前面的模式零次或多次。
  \item +:匹配前面的模式一次或多次。
  \item?:匹配前面的模式零次或一次。
  \item \{n\}:匹配前面的模式恰好 n 次。
  \item \{n,\}:匹配前面的模式至少 n 次。
  \item \{n,m\}:匹配前面的模式至少 n 次且不超过 m 次。
  \item [[a-z]]:匹配括号内的任意一个字符。例如,[abc] 匹配字符 "a"、"b" 或 "c"。 
  \item 课后作业小练习:匹配你的 QQ号 \\ \texttt{[0-9]+\{5,11\}}
\end{itemize}
\end{document}
