\documentclass{article}
\usepackage{xcolor}
\usepackage{graphicx}
\usepackage[UTF8]{ctex}
\usepackage{listings}  
\lstset{  
    basicstyle=\ttfamily,  
    numbers=left,  
    numberstyle=\small,  
    frame=single,  
    showstringspaces=false,  
    language=Python  
}  
\begin{document}
\title{中国海洋大学实验报告}
\maketitle
\title{shell,vim,数据整理的学习\\https://github.com/YAYaun/2024\\}
\author{袁东霖}
\maketitle
\section{\textcolor{blue}{Creating and Navigating Directories}}
\begin{itemize}
    \item \texttt{mkdir -p "name"}:创建并打开 \texttt{name},如果没有父目录则一并创建。
    \item \texttt{cd "name"}:进入 \texttt{name} 目录。
\end{itemize}

\section{\textcolor{blue}{Viewing Help Documentation}}
\begin{itemize}
    \item \texttt{man 函数名}:查看函数的帮助文档。
\end{itemize}

\section{\textcolor{blue}{Viewing Files(课后习题)}}
\begin{itemize}
    \item \texttt{ls -a}:查看所有文件,包含隐藏文件。
    \item \texttt{ls -lah}:以长文件形式显示,以人类可读的形式显示。
    \item \texttt{ls -lt}:显示最新修改的文件。
    \item \texttt{ls -lu}:显示最新访问的文件。
\end{itemize}{{Battery Information}}
\begin{itemize}
    \item \texttt{cat /sys/class/power\_supply/BAT0/capacity}:获取电池信息。
\end{itemize}
\section{\textcolor{blue}{Shell基本知识}}
返回值短路
\begin{lstlisting}
false || echo "Oops, fail"
# Oops, fail
true || echo "Will not be printed"
#
true && echo "Things went well"
# Things went well
false && echo "Will not be printed"
#
false ; echo "This will always run"
# This will always run
\end{lstlisting}
\section{\textcolor{blue}{shell脚本的通配操作}}
\begin{lstlisting}  
# convert image.{png,jpg}   
convert image.png image.jpg  
\end{lstlisting}  
  
\section{\textcolor{blue}{Marco and Polo Functions(课后习题)}}
\begin{itemize}
    \item \texttt{marco()\{ export marcoPath=\$(pwd); \}}:定义 \texttt{marco} 函数,保存当前路径。
    \item \texttt{polo()\{ cd "\$marcoPath"; \}}:定义 \texttt{polo} 函数,返回 \texttt{marco} 保存的路径。
\end{itemize}
\section{\textcolor{blue}{Shell时间操作 }}
    \verb+echo "Starting program at $(date)" # date会被替换成日期和时间 + 
\section{\textcolor{blue}{Shell输入参数倒序输出}}
\begin{lstlisting}
import sys
for arg in reversed(sys.argv[1:]):
    print(arg)
\end{lstlisting}
\section{\textcolor{blue}{Shell查找文件}}
\begin{lstlisting}
find . -name src -type d
find . -path '*/test/*.py' -type f
find . -mtime -1
find . -size +500k -si
\end{lstlisting}
\section{\textcolor{blue}{Shell简单的替代和删除操作}}
\begin{lstlisting}
find . -name '*.tmp' -exec rm {} \;
find . -name '*.png' -exec convert {} {}.jpg \;
\end{lstlisting}
\section{\textcolor{blue}{Vim Commands}}
\begin{itemize}
\begin{figure}[htbp]
    \centering
    \includegraphics[scale=0.2]{picturevim.png}
    \caption{Elliptic Paraboloid}
    \end{figure} 
    \item \texttt{i}:切换到输入模式。
    \item \texttt{a}:切换到插入模式。
    \item \texttt{x}:删除当前光标所在字符。
    \item \texttt{u}:撤销上一次操作。
    \item \texttt{ctrl+r}:重做上一次的操作。
    \item \texttt{:wq}:保存并退出。
    \item \texttt{\textbackslash sudo vim 文件名}:以管理员权限打开文件。
    \item \texttt{:s/foo/bar/g}:在整个文件把 \texttt{foo} 改成 \texttt{bar}。
\end{itemize}
\section{\textcolor{blue}{Regular expression}}
\begin{itemize}
  \item *:匹配前面的模式零次或多次。
  \item +:匹配前面的模式一次或多次。
  \item?:匹配前面的模式零次或一次。
  \item \{n\}:匹配前面的模式恰好 n 次。
  \item \{n,\}:匹配前面的模式至少 n 次。
  \item \{n,m\}:匹配前面的模式至少 n 次且不超过 m 次。
  \item [[a-z]]:匹配括号内的任意一个字符。例如,[abc] 匹配字符 "a"、"b" 或 "c"。 
  \item 课后作业小练习:匹配你的 QQ号 \\ \texttt{[0-9]+\{5,11\}}
\end{itemize}

\section{\textcolor{blue}{正则练习1}}
\begin{itemize}
\item \verb|...\.| \\ 匹配cat. 896. ?=+. \\skip abc1
\end{itemize}
\section{\textcolor{blue}{正则练习2}}
\begin{itemize}
\item \verb|[cmf]an| \\ 匹配can fan man \\skip dan ran pan
\end{itemize}
\section{\textcolor{blue}{正则练习3}}
\begin{itemize}
\item \verb|[^b]og| \\ 匹配hog dog not bog
\end{itemize}
\section{\textcolor{blue}{正则练习4}}
\begin{itemize}
\item \verb|[A-C][n-p][a-c]| \\ 匹配 Ana Bob Cpc \\skip aax bby ccz
\end{itemize}
\section{\textcolor{blue}{正则练习5}}
\begin{itemize}
\item \verb|waz{3,5}up| \\ 匹配wazzzzzup wazzzup \\skip wazup
\end{itemize}
\section{\textcolor{blue}{正则练习6}}
\begin{itemize}
\item \verb|a{2,4}b{0,4}c{1,2}| \\ 匹配aaaabcc aabbbbc aacc skip a
\end{itemize}
\section{\textcolor{blue}{正则练习7}}
\begin{itemize}
\item \verb|\d+ files? found\?| \\ 匹配1 file found?  2 files found? 24 files found?
\\ skip No files found.
\end{itemize}
\section{\textcolor{blue}{正则练习8}}
\begin{itemize}
\item \verb|\d\.\s+abc| \\ 匹配 " abc" \\skip"abc"
\end{itemize}
\section{\textcolor{blue}{git记录}}
\begin{lstlisting}
$ git push -u origin main
Enumerating objects: 9, done.
Counting objects: 100% (9/9), done.
Delta compression using up to 20 threads
Compressing objects: 100% (7/7), done.
Writing objects: 100% (7/7), 2.79 KiB | 2.79 MiB/s, done.
Total 7 (delta 2), reused 0 (delta 0), pack-reused 0 (from 0)
remote: Resolving deltas: 100% (2/2), done.
To github.com:YAYaun/2024.git
   6e418e1..5035f2d  main -> main
branch 'main' set up to track 'origin/main'.                                                                                                                    
\end{lstlisting}
\end{document}
