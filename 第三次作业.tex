\documentclass{article}  
\usepackage{xcolor}  
\usepackage{listings}  
\usepackage[UTF8]{ctex}  
  
\lstset{  
    basicstyle=\ttfamily,  
    numbers=left,  
    numberstyle=\small,  
    frame=single,  
    showstringspaces=false,  
    language=Python  
}  
  
\begin{document}  
\title{\textcolor{red}{中国海洋大学实验报告}\\命令行环境, Python入门基础, Python视觉应用}  
\author{袁东霖}  
\date{2024/9/6}  
\maketitle  
https://github.com/YAYaun/2024 \\
\section{\textcolor{blue}{命令行环境}}  
命令行环境课后作业1 \\  
结束所有的 sleep 命令而不用 pid: \texttt{pkill -af sleep} \\  
如何在某个进程结束后再进行指定的命令:  
\begin{lstlisting}  
pidwait() {    
  while kill -0 $1; do  
    sleep 1    
  done    
  ls    
}  
\end{lstlisting}  
别名: \\  
\begin{lstlisting}
\texttt{alias dc='cd'} \\
\end{lstlisting}  
这样我们错误的将 cd 输入为 dc 时也能正确执行 \\  
\section{\textcolor{blue}{命令行结束进程}}
\begin{lstlisting}
import signal, time
def handler(signum, time):
    print("\nI got a SIGINT, but I am not stopping")
signal.signal(signal.SIGINT, handler)
i = 0
while True:
    time.sleep(.1)
    print("\r{}".format(i), end="")
    i += 1
\end{lstlisting}
\section{\textcolor{blue}{ssh配置远端设备使用}}
远端设备使用:  
\begin{lstlisting}  
ssh-keygen -o -a 100 -t ed25519 -f ~/.ssh/id_ed25519
\end{lstlisting}  
创建一个 ssh 密钥对 \\  
.ssh/config 中添加内容:  
\begin{lstlisting}  
Host vm  
  User username_goes_here  
  HostName ip_goes_here  
  IdentityFile ~/.ssh/id_name  
  LocalForward 9999 localhost:8888  
\end{lstlisting}  
\texttt{ssh-copy-id vm} 将自己的 ssh 密钥拷贝到服务器 \\  
之后使用 \texttt{ssh vm} 就可以进行免密登录  
\\ 接下来是需要拷贝大量文件时可以用scp命令遍历相关路径
\begin{lstlisting}
scp path/to/local_file remote_host:path/to/remote_file
\end{lstlisting}
rsync对scp进行了相关改进,还可以基于--partial标记实现断电续传
\section{\textcolor{blue}{终端多路复用}}
\verb+tmux 开始一个新的会话+
\verb+tmux new -s NAME 以指定名称开始一个新的会话+\\
\verb+tmux ls 列出当前所有会话+\\
\verb+在 tmux 中输入 <C-b> d ,将当前会话分离+\\
\verb+tmux a 重新连接最后一个会话。您也可以通过 -t 来指定具体的会话+\\
\section{\textcolor{blue}{Python简介}}
Python 是一种解释型语言: 这意味着开发过程中没有了编译这个环节。\\
Python 是交互式语言: 这意味着,您可以在一个 Python 提示符 >>> 后直接执行代码。\\
Python 是面向对象语言: 这意味着Python支持面向对象的风格或代码封装在对象的编程技术
\section{\textcolor{blue}{Python第三方库的安装}}
在pycharm中可以通过镜像网站安装,pycharm官方网址太慢
\section{\textcolor{blue}{Python正则表达式}}
来自菜鸟教程
\begin{lstlisting}
import re
print(re.match('www', 'www.runoob.com').span())
print(re.match('com', 'www.runoob.com'))
\end{lstlisting}
\section{\textcolor{blue}{Python循环}}
\begin{lstlisting}
while condition:
    \texttt{these are your codes}
for number in sequences:
    \texttt{these are your codes}
\end{lstlisting}
\section{\textcolor{blue}{Python元组}}
无法被改变
\begin{lstlisting}
tup1 = ('1924', '10', 25, 2005)
tup2 = (1, 2, 3, 4, 5 )
tup3 = "a", "b", "c", "d"
\end{lstlisting}
\section{\textcolor{blue}{Python字典}}
\begin{lstlisting}
>>> dict = {'a': 1, 'b': 2, 'b': '3'}
>>> dict['b']
'3'
>>> dict
{'a': 1, 'b': '3'}  
\end{lstlisting}
\section{\textcolor{blue}{Python质数判断}}  
利用基本知识写的判断质数:  
\begin{lstlisting}  
def isprime(n):
    if n < 2:
        return False
    for i in range(2, int(n ** 0.5) + 1):
        if n % i == 0:
            return False
    return True
n = int(input("int please: "))
print(isprime(n))  
\end{lstlisting}  
\section{\textcolor{blue}{Python基本语法}}
\begin{itemize}
\item if elif
\item while:
\item def function(peremeter):
\item 以缩进为语块
\end{itemize}
\section{\textcolor{blue}{Python面向对象}}
\begin{lstlisting}
class Animal:
   age = 0
   def __init__(age, name):
      self.name = name
      self.age = age56dddsaxfttffffff
\end{lstlisting}
\section{\textcolor{blue}{Python三维画图}}
\begin{lstlisting}
import numpy as np
z = np.linspace(0,13,1000)
x = 5*np.sin(z)
y = 5*np.cos(z)
zd = 13*np.random.random(100)
xd = 5*np.sin(zd)
yd = 5*np.cos(zd)
ax1.scatter3D(xd,yd,zd, cmap='Blues') 
ax1.plot3D(x,y,z,'gray') 
plt.show()
\end{lstlisting}
\section{\textcolor{blue}{Python等高线绘图}}
\begin{lstlisting}
from matplotlib import pyplot as plt
from mpl_toolkits.mplot3d import Axes3D
fig4 = plt.figure()
ax4 = plt.axes(projection='3d')
xx = np.arange(-5,5,0.1)
yy = np.arange(-5,5,0.1)
X, Y = np.meshgrid(xx, yy)
Z = np.sin(np.sqrt(X**2+Y**2))
ax4.contour(X,Y,Z,zdir='z', offset=-3,cmap="rainbow")  
ax4.contour(X,Y,Z,zdir='x', offset=-6,cmap="rainbow") 
ax4.contour(X,Y,Z,zdir='y', offset=6,cmap="rainbow")   
#ax4.contourf(X,Y,Z,zdir='y', offset=6,cmap="rainbow")  
ax4.set_xlabel('X')
ax4.set_xlim(-6, 4) 
ax4.set_ylabel('Y')
ax4.set_ylim(-4, 6)
ax4.set_zlabel('Z')
ax4.set_zlim(-3, 3)
plt.show()
\end{lstlisting}
\section{\textcolor{blue}{PythonJit}}
\begin{itemize}
\item jit可以加速数学计算
\item jit并不能加快所有代码
\item 函数加jit时是不会被编译报错的,需要先调试后再加jit
\end{itemize}
\section{\textcolor{blue}{Python图像数组化}}
\begin{itemize}
\item 图像数组化\\
\begin{lstlisting}
from PIL import Image
from pylab import *
im = array(Image.open(r'C:\Users\11378\Pictures\picture.jpg'))
imshow(im)
x = [100,100,400,400]
y = [200,500,200,500]
plot(x,y,'r*')
plot(x[:2],y[:2])
title('Plotting: "empire.jpg"')
show()
\end{lstlisting}
\end{itemize}
\section{\textcolor{blue}{Python绘制散点图}}
\begin{itemize}
\item 绘制散点图
\begin{lstlisting}  
import matplotlib.pyplot as plt  
x = [1, 2, 3, 4, 5, 6, 7, 8, 9, 10]  
y = [2, 3, 5, 7, 8, 5, 6, 9, 11, 5] 
  
plt.scatter(x, y)
  
plt.xlabel('X')    
plt.ylabel('Y')  
plt.title('title')  
plt.show()
\end{lstlisting}
\end{itemize}
\section{\textcolor{blue}{Python图像导数灰暗显示}}
\begin{lstlisting}
from PIL import Image
 from numpy import *
 from scipy.ndimage import filters
 im = array(Image.open('empire.jpg').convert('L'))
imx = zeros(im.shape)
 filters.sobel(im,1,imx)
 imy = zeros(im.shape)
 filters.sobel(im,0,imy)
 magnitude = sqrt(imx**2+imy**2)
\end{lstlisting}
\section{\textcolor{blue}{Python课后作业}} 
\begin{itemize}
\item 下面是课后作业,用高斯模糊后,原数组减去高斯模糊的数组
\begin{lstlisting}  
from PIL import Image, ImageFilter  
image = Image.open(r'C:\Users\11378\Pictures\picture.jpg')  
blurred_image = image.filter(ImageFilter.GaussianBlur(radius=5))  
width, height = image.size  
image_diff = Image.new('RGB', (width, height))  
for x in range(width):  
    for y in range(height):  
        original_pixel = image.getpixel((x, y))  
        blurred_pixel = blurred_image.getpixel((x, y))  
        diff_pixel = (  
            abs(original_pixel[0] - blurred_pixel[0]),  
            abs(original_pixel[1] - blurred_pixel[1]),  
            abs(original_pixel[2] - blurred_pixel[2])  
        )  
        image_diff.putpixel((x, y), diff_pixel)  
image_diff.show()  
image_diff.save('path_to_save_diff_image.jpg')
\end{lstlisting}   
\end{itemize}
\end{document} 